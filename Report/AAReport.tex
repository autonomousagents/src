\documentclass{article}

\usepackage[margin=1in]{geometry}
\usepackage{amsmath,amssymb}


\begin{document}

\title{Autonomous Agents Report}
\author{Agnes van Belle, Maaike Fleuren, Norbert Heijne, Lydia Mennes}
\maketitle

This report has been written for the Master Artificial Intelligence course Autonomous Agents. The assignments that will be worked out in detail in this report include the following topics: "Single Agent Planning, Single Agent Learning, and Multi-Agent Planning and Learning". In this report motivation will be given for the design choices and specific programming choices that have been made. The explanation and motivation follows the must-have (depicted by \textbf{M}) and should/could-have (depicted by \textbf{SC}) structure that has been set up for the assignments. By doing so we hope to make our report easy to read for the teachers/assistants that will be grading this.

\section{Assignment 1: Single Agent Planning}

\subsection{\textbf{M} Simulating the Environment}
The choice has been made to not encode the positions of the agents as part of a grid, i.e. matrix. Instead the agents both know their own position and on each iteration the position of the other agent is given as input by the environment, as we have stated in the Agent interface. This is needed for prey to see if the predator is next to it, to prevent it moving towards the prey. In the predator's case it might be necessary later on to know the position of the prey although it is not necessary for this particular sub-assignment.

As part of the assignment a mean and a standard deviation was asked for 100 runs with the use of the random policy for the predator's behaviour. The lowest amount of time steps observed was 19 time steps. The optimal amount of time steps given that the prey would remain still throughout the trial run would be 10. The highest amount observed was 1194 steps. The average amount of time steps was 296.93 time steps and the standard deviation was 244.55 time steps (rounded up). This is a clear indication of the inefficiency of the random policy in this particular setting. 

\section{Appendix}
Assignment 1 first \textbf{M}, 100 runs for the random policy predator.\\\\
Timesteps:421\\
Timesteps:831\\
Timesteps:476\\
Timesteps:74\\
Timesteps:537\\
Timesteps:40\\
Timesteps:468\\
Timesteps:465\\
Timesteps:105\\
Timesteps:123\\
Timesteps:227\\
Timesteps:658\\
Timesteps:696\\
Timesteps:153\\
Timesteps:426\\
Timesteps:431\\
Timesteps:24\\
Timesteps:197\\
Timesteps:517\\
Timesteps:313\\
Timesteps:492\\
Timesteps:213\\
Timesteps:457\\
Timesteps:392\\
Timesteps:47\\
Timesteps:178\\
Timesteps:459\\
Timesteps:624\\
Timesteps:881\\
Timesteps:100\\
Timesteps:244\\
Timesteps:127\\
Timesteps:213\\
Timesteps:145\\
Timesteps:45\\
Timesteps:301\\
Timesteps:628\\
Timesteps:248\\
Timesteps:88\\
Timesteps:123\\
Timesteps:82\\
Timesteps:206\\
Timesteps:181\\
Timesteps:771\\
Timesteps:114\\
Timesteps:238\\
Timesteps:118\\
Timesteps:67\\
Timesteps:41\\
Timesteps:662\\
Timesteps:27\\
Timesteps:73\\
Timesteps:217\\
Timesteps:269\\
Timesteps:382\\
Timesteps:60\\
Timesteps:205\\
Timesteps:64\\
Timesteps:133\\
Timesteps:232\\
Timesteps:148\\
Timesteps:504\\
Timesteps:113\\
Timesteps:316\\
Timesteps:151\\
Timesteps:178\\
Timesteps:53\\
Timesteps:526\\
Timesteps:150\\
Timesteps:690\\
Timesteps:490\\
Timesteps:116\\
Timesteps:288\\
Timesteps:79\\
Timesteps:163\\
Timesteps:266\\
Timesteps:566\\
Timesteps:1194\\
Timesteps:133\\
Timesteps:690\\
Timesteps:136\\
Timesteps:121\\
Timesteps:123\\
Timesteps:492\\
Timesteps:288\\
Timesteps:185\\
Timesteps:19\\
Timesteps:78\\
Timesteps:250\\
Timesteps:42\\
Timesteps:268\\
Timesteps:190\\
Timesteps:231\\
Timesteps:393\\
Timesteps:338\\
Timesteps:100\\
Timesteps:49\\
Timesteps:653\\
Timesteps:1173\\
Timesteps:421\\
Average timesteps over 100 trials: 296.93\\
Standard deviation over 100 trials: 244.54689754728025\\

\end{document}
